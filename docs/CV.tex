%%%%%%%%%%%%%%%%%%%%%%%%%%%%%%%%%%%%%%%%%
% Medium Length Professional CV
% LaTeX Template
% Version 2.0 (8/5/13)
%
% This template has been downloaded from:
% http://www.LaTeXTemplates.com
%
% Original author:
% Trey Hunner (http://www.treyhunner.com/)
%
% Important note:
% This template requires the resume.cls file to be in the same directory as the
% .tex file. The resume.cls file provides the resume style used for structuring the
% document.
%
%%%%%%%%%%%%%%%%%%%%%%%%%%%%%%%%%%%%%%%%%

%----------------------------------------------------------------------------------------
%	PACKAGES AND OTHER DOCUMENT CONFIGURATIONS
%----------------------------------------------------------------------------------------

\documentclass{resume} % Use the custom resume.cls style
\usepackage{fontawesome5}
\usepackage[backref]{hyperref} 
\hypersetup
{   pdfauthor = UnnamedOrange,
	colorlinks = true,
	linkcolor = black,
	anchorcolor = black,
	citecolor = black,
	urlcolor = black
}


\usepackage[left=0.75in,top=0.6in,right=0.75in,bottom=0.6in]{geometry} % Document margins

\name{Yanglang Yuan} % Your name
\address{1088 Xueyuan Avenue, Shenzhen 518055, P.R. China} % Your address
\address{ {\href{mailto:11930509@mail.sustech.edu.cn}{\raisebox{-0.2\height}\faEnvelope\  {11930509@mail.sustech.edu.cn}}}{~~
\href{https://www.linkedin.com/in/max-yuan-180bb7239}{\raisebox{-0.2\height}\faLinkedin\ {linkedin.com/Max Yuan}}}
{~~\href{https://yanglangyuan.github.io}{\faGithub\ {yanglangyuan.github.io}}}~~{\faCalendar\ {May. 2022}} }


\begin{document}

%----------------------------------------------------------------------------------------
%	EDUCATION SECTION
%----------------------------------------------------------------------------------------

\begin{rSection}{Education}

{\bf Southern University of Science and Technology} \hfill {\em Sept. 2021 - June 2023(Expected)} \\ 
{M.Sc. in Electronic Science and Technology} \hfill {Shenzhen, China}

{\bf Southern University of Science and Technology} \hfill {\em Sept. 2015 - June 2019} \\ 
{B.E. in Mechanical Engineering} \hfill {Shenzhen, China}

\end{rSection}

%----------------------------------------------------------------------------------------
%	PUBLICATIONS
%----------------------------------------------------------------------------------------

\begin{rSection}{Publications \& patents}
   % \begin{rSubsection}{Journal Papers}{}{}{}

        % \item {J2 \textbf{C.Zhu}, J. Grezmak, B. Schmidt, K. Daltorio(2022), “A Dactyl-Integrated Sensor Design for Measuring Lake Waves", \textit{Soft Robotics, Brief Communication(Submitted)}} \\
        % \vspace{-10pt}
%   \end{rSubsection}

      %  \begin{rSubsection}{Journal Papers}{}{}{}
      %   \item {\textbf{Yanglang Yuan}, J. Grezmak, Yiming Rong, K. Daltorio(2022), “A Review of Individual Customization in Manufacturing System", 
       %  \textit{Soft Robotics, Brif Communication(Published)}} \\
        % \vspace{-10pt}


       %  \item {J1 J. Zhou, Q. Nguyen, S. Kamath, Y. Hacohen, \textbf{C.Zhu}, M. Fu, K. Daltorio(2022), “Hands to Hexapods, Wearable User Interface Design for Specifying Leg Placement for Legged Robots", \textit{Frontiers in Robotics and AI(Submitted)}}
       % \end{rSubsection}
       % \vspace{-8pt}
    % \begin{rSubsection}{Conference Papers}{}{}{}
    %     \item {C1 \textbf{C.Zhu}, F. Han, J. Yi,  “Wearable Sensing and Knee Exoskeleton Control for Awkward Gaits Assistance", \textit{2022 IEEE 18th International Conference on Automation Science and Engineering Mexico City, Mexico(Submitted)}}
    % \end{rSubsection}
    
       \begin{rSubsection}{Patents}{}{}{}
       \item { \textbf{Yanglang Yuan}, Jiahui Zhang, Xiaoyu Zhang, Fan Zhou, Yajun Wang, Hui Li, Haijiang Wang. Fuel cell bipolar plate and fuel cell: CN, 110459780A[P]. 2019-11-05.}
       \vspace{2pt}
       
       \item { \textbf{Yanglang Yuan}, Jiahui Zhang, Xiaoyu Zhang, Fan Zhou, Yajun Wang, Hui Li, Haijiang Wang. Fuel cell bipolar plate and fuel cell: CN, 210429963U[P]. 2019-07-25.}
       
       
       \end{rSubsection} 
    
            
\end{rSection}

%----------------------------------------------------------------------------------------
%	RESEARCH EXPERIENCE SECTION
%----------------------------------------------------------------------------------------

\begin{rSection}{Research Experiences}

    \begin{rSubsection}{Intelligent Manufacturing Lab at SUSTech}{July 2021 - Present}{Advisor: Prof. Yiming Rong}{Shenzhen, China}
    \vspace{-4pt}
    \item \textbf{Researching for Customer-Manufacturer(C2M) } \\
    Data driven intelligent manufacturing system;\\
    C2M manufacturing system model with universality and completeness;\\
    3D printing sneakers based on C2M scene.
    \end{rSubsection}
    
    %------------------------------------------------
    
    \begin{rSubsection}{Shenzhen Key Laboratory of Hydrogen Energy}{June 2017 - June 2019}{Advisor: Prof. Haijiang Wang}{Shenzhen, China}
    \vspace{-4pt}
    \item \textbf{A Bipolar Plate Design for Fuel Cell} \\
    Mechanical structure design of a novel bipolar plate for fuel cell unmanned aerial vehicle (UAV);\\
    Simulated in SolidWorks and COMSOL for the static and dynamic analysis.

    \item \textbf{A Controlling System for Fuel Cell}\\ 
    Designed the control logic and methods \& controller of the fuel cell system; \\
    Designed the temperature and voltage patrol detection system of Fuel Cell.\\
    Thesis topic: Controlling System of the fuel cell in unmanned aerial vehicle (UAV)
%    \item \textbf{Geotechnical Modeling for CRAB-Like Robot Locomotion on Granular Medias}\\
 %   Built hexapod robot model in Webots simulator and developed a tripod gait controller;\\
  %  Designed experiments to derive the leg-terrain interaction model based on Resistive Force Theory;\\
   % Proposed a way to reduce the overlapping time of stance for different gaits and minimize the sinkage.

    \end{rSubsection}

    \begin{rSubsection}{Seoul National University - SUSTech Research Program}{Dec. 2017 - Jan. 2018}{Advisor: Prof. Keyang Tang}{Shenzhen,China}
        \vspace{-4pt}
        \item \textbf{International Winter Undergraduate Research Experience} \\
        Worked in mixed teams with Seoul National University students to complete specific research tasks;\\ 
        Analysis on the development of intelligent manufacturing industry; \\
        Brief hypothesis of integrated system for design, manufacture and maintenance of shared bicycle.
    \end{rSubsection}

    \begin{rSubsection}{Robotics and Autonomy Lab at Tsinghua University}{June 2017 - Aug. 2017}{Advisor: Prof. Chenglong Fu}{Beijing, China}
        \vspace{-4pt}
        \item \textbf{Summer Undergraduate Research Project} \\
        Participated in the design and manufacturing of robot Bluetooth communication controller; \\
        Other advanced knowledge in Robotics.
    \end{rSubsection}

    \begin{rSubsection}{Formula Student Electric China Program}{2016 - 2017}{Advisor: Prof. Yiming Rong}{Shenzhen,China}
        \vspace{-4pt}
        \item \textbf{The Controlling System of Electric Racing}  \hfill {\em Jan. 2017 - Aug. 2017}\\
        Designed the control logic and methods of the electric racing;\\
        Built the high voltage and low voltage circuits of the racing.
        \item \textbf{Battery Management System(BMS) of Electric Racing}  \hfill {\em June 2016 - Jan. 2017}\\
        Research on control strategy of battery management system; \\
        Designed and built the temperature and voltage detection circuit of battery management system.

    \end{rSubsection}
 
\end{rSection}
%----------------------------------------------------------------------------------------
%	TEACHING EXPERIENCE SECTION
%----------------------------------------------------------------------------------------

\begin{rSection}{Teaching Experience}

\begin{rSubsection}{Advanced Manufacturing Systems}{Feb. 2022 - June 2022}{Teaching Assistant}{Shenzhen, China}
\item The primary goal of this course is to impart to the student an understanding of advanced systems for the production of mechanical components using the latest technologies and methods and to enable the student to analyze systems for the production of mechanical components using modern advanced processes and technologies.Discussions are presented related to the system integration of computer-aided design (CAD), computer-aided engineering (CAE), computer-aided manufacturing (CAM), robotics, material resource planning, tool management, information management, process control, quality control, etc.

\end{rSubsection}

%------------------------------------------------

\begin{rSubsection}{Awareness Practices of Manufacturing Engineering}{Sept. 2021 - Jan. 2022}{Teaching Assistant}{Shenzhen, China}
\item This is an awareness practice course to gain some basic knowledge of fundamental manufacturing principle and methods through learning and operating typical manufacturing equipment, such as Numerical Controlled (NC) machine tools, measurement devices, additive manufacturing (3D printing) machines, etc. Students are expected to establish an understanding of basic manufacturing methods and operations as well as the concept of quality. It is a foundation for further learning in related engineering disciplines. 
\item Responsible for preparation of equipments, data analysis, and presentation of results and conclusions of the students' course projects.
\end{rSubsection}

\end{rSection}

\vspace{10pt}
%----------------------------------------------------------------------------------------
%	TECHNICAL STRENGTHS SECTION
%----------------------------------------------------------------------------------------

\begin{rSection}{Skills}

\begin{tabular}{ @{} >{\bfseries}l @{\hspace{6ex}} l }
Certificate & GCDF \\
Programming & Proficient: MATLAB, Plant Simulation;   ~~Intermediate: C/C++, Java \\
Technical Tools & LATEX, Solidworks, 3D-printing, CAM, Altium Designer, etc \\
Language & Chinese, Cantonese, English
\end{tabular}

\end{rSection}


%----------------------------------------------------------------------------------------
%	THONORS AND AWARDS
%----------------------------------------------------------------------------------------

\begin{rSection}{Honors and Awards}
    \vspace{-8pt}
    \item {Excellent Teaching Assistant, SUSTech}   \hfill {2022}  
     \vspace{-5pt}
    \item {Employee of the Year, SUSTech}   \hfill {2021}  
     \vspace{-5pt}
    \item {Outstanding Graduation Thesis, SUSTech}        \hfill {2019}
     \vspace{-5pt}
    \item {First-Class Honors Graduate of Zhicheng Residential College, SUSTech}   \hfill {2019}
     \vspace{-5pt}
    \item {Self-improvement Star of Zhicheng Residential College, SUSTech}   \hfill {2019}
     \vspace{-5pt}
    \item {National Encouragement Scholarship, China}        \hfill {2018}
     \vspace{-5pt}
    \item {Outstanding Leadership of Zhicheng Residential College, SUSTech} \hfill {2016 - 2018}
    \vspace{-5pt}  
     \item {Winning Award of Formula Student Electric China} \hfill {2017}
    \vspace{-5pt}
%     \item {Winning Award of Formula Student Electric China} \hfill {2017}
 %   \vspace{-5pt}
%   \item {Outstanding Student Representative of Student Congress, SUSTech} \hfill {2017}

%  \vspace{-5pt}  
%    \item {Outstanding Student Representative of Student Congress, SUSTech}    \hfill {2017}
     


%    \item {First-Class SUSTech Scholarship for Outstanding Students}        \hfill {2018}  Academic progress award
%    \vspace{-5pt}
%    \item {Third-Class SUSTech Scholarship for Outstanding Student}         \hfill {2015-2017}
%    \vspace{-5pt}    
    \end{rSection}

%----------------------------------------------------------------------------------------
%	EXAMPLE SECTION
%----------------------------------------------------------------------------------------

%\begin{rSection}{Section Name}

%Section content\ldots

%\end{rSection}

%----------------------------------------------------------------------------------------

\end{document}
